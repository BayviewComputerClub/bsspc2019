\documentclass[]{article}
\usepackage{graphicx}
\usepackage[normalem]{ulem}
\usepackage[a4paper, total={6in, 8in}]{geometry}
\setlength\parindent{0pt}
\setlength{\parskip}{1em}
\graphicspath{ {./} }
\thispagestyle{empty}

\begin{document}

{\Large BSSPC ‘19 P4 - Katyusha!}
\\\\
\textbf{\large Problem Statement}
\par
It is the year $1945$. You are a Soviet \textit{Super-Soldier} and can predict the future. You have just completed a mission to sabotage the dying capital of the Third Reich. You need to safely return to your \textit{comrades }advancing on the outskirts of Berlin.

\par
However, there are several \textbf{Katyusha} units that are currently firing on the area between you and your comrades. There are three types of \textbf{Katyushas}: the light \textit{BM-8}, with an impact radius of $8$ meters, the intermediate \textit{BM-13}, with an impact radius of $13$ meters, and the massive \textit{BM-30}, with an impact radius of a whopping $30$ meters. Comrade Super-Soldier, you need to determine whether you can return to your comrades and complete the final assault in the Great Patriotic War!

\par
There will be a grid of $101$ by $101$ squares. Each of the $N$ \textbf{Katyusha} units will turn a square at $(x,y)$ into a death-zone (\textbf{DZ}). You start from the bottom left corner (square $(0,0)$) of the grid and must reach the top right corner (square $(100,100)$) of the grid without entering any \textbf{DZ}. You may travel in any direction laterally, but may not travel diagonally. Good luck, \textit{comrade}!

\textbf{\large Input Specification}
\par
The first line will contain one integer, $0 \le N \le 5000$. Each of the next $N$ lines will contain $2$ space-separated integers, $0 \le x_i \le 100$ and $0 \le y_i \le 100$.

\textbf{\large Output Specification}
\par
You will print a single character, $y$ to indicate that you can reach your comrades or $n$ to indicate that you will die for the motherland.
\\\\
\textbf{\large Sample Input 1}
\begin{verbatim}
2
0 1
1 0
\end{verbatim}
\textbf{\large Sample Output 1}
\begin{verbatim}
n
\end{verbatim}

\textbf{\large Sample Input 2}
\begin{verbatim}
21
0 1
1 1
2 1
4 0
4 1
4 2
4 3
3 3
2 3
1 3
0 7
1 7
2 7
3 7
4 7
5 7
6 7
6 6
6 5
6 4
6 3
\end{verbatim}
\textbf{\large Sample Output 2}
\begin{verbatim}
y
\end{verbatim}

\thispagestyle{empty}
\end{document}